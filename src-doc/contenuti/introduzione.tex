\chapter{Introduzione}

\section{Scopo del Documento}
Nel seguente documento viene illustrato in modo dettagliato la struttura e la composizione dei vari microservizi che compongo il prodotto. Il documento tratterà, in ordine, i seguenti punti:
\begin{itemize}
    \item WebApp;
    \item sistema di logging;
    \item sistema di illuminazione;
    \item sistema di anagrafica;
    \item sistema di autorizzazione.
\end{itemize}

\section{Scopo dell'architettura}
L'obiettivo di SWEasabi e dell'azienda ImolaInformatica S.p.A. è lo sviluppo di un sistema per l'ottimizzazione dell'illuminazione, il prodotto presenta differenti servizi che comunicano tra loro, ogni servizio ha un compito ben preciso e si occupa di una parte del sistema per questo necessità di una architettura ben definita e strutturata.
In questo documento verrà presentata l'architettura dei vari sistemi, i design pattern utilizzati e le tecnologie adottate.


\section{Glossario}
Per evitare ambiguità relative alle terminologie utilizzate è stato creato un documento denominato \href{https://github.com/SWEasabi/glossario/releases}{Glossario}.

Questo documento contiene tutti i termini specifici di settore utilizzati nei documenti, con le relative definizioni.

\section{Maturità del documento}
Il presente documento ha raggiunto un buon grado di maturità, in quanto sono state definite le tecnologie e i design pattern utilizzati per lo sviluppo del prodotto. Inoltre sono state definite le interazioni tra i vari servizi che compongono il prodotto.

\section{Riferimenti}
\subsection{Riferimenti Normativi}
\begin{itemize}
    \item \href{https://github.com/SWEasabi/norme-di-progetto/releases}{Norme di progetto};
    \item \href{https://www.math.unipd.it/~tullio/IS-1/2022/Progetto/C2.pdf}{capitolato d'appalto C2}.
\end{itemize}

\subsection{Riferimenti Informativi}
\begin{itemize}
    \item \href{https://github.com/SWEasabi/analisi-dei-requisiti/releases}{Analisi dei requisiti}
    \item \href{https://www.math.unipd.it/~rcardin/swea/2022/Software%20Architecture%20Patterns.pdf}{Software Architecture Patterns};
    \item \href{https://jwt.io/}{JSON Web Token website};
    \item \href{https://dev.to/dyarleniber/hexagonal-architecture-and-clean-architecture-with-examples-48oi}{clean architecture with examples};
\end{itemize}