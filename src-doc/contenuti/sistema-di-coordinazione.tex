\chapter{Sistema di coordinazione}\label{cap:sistema-coordinazione}

\section{Scopo del sistema}
Il sistema di coordinazione è il componente che si occupa di gestire i lampioni, regolandone l'accensione e lo spegnimento in base alle informazioni che ha a disposizione.\\

Tramite algoritmi sofisticati, il sistema di coordinazione è in grado di gestire l'illuminazione in modo efficiente.

\subsection{Requisiti coperti dal sistema}

\subsubsection{RF\_05}

RF\_05: Il sistema deve accendere un'area per un lasso di tempo preconfigurato quando rileva persone in prossimità dello stesso.

\subsubsection{RF\_06}

RF\_06: Il sistema deve riportare l'intensità luminosa dell'area al valore di default una volta passato il tempo impostato.

\subsubsection{RF\_19}

RF\_19: Il sistema deve essere in grado di ricevere informazioni dal sensore in modalità push.

\section{Descrizione del sistema}

Il sistema utilizza molteplici paradigmi. L'architettura generale è di tipo esagonale.

Il programma eseguirà azioni quando attivato da eventi esterni, quali l'arrivo di un nuovo stato da mqtt oppure l'arrivo di una richiesta da parte dell'utente collegato alla webapp\footnote{Vedi sezione \ref{cap:webapp}}.

Le componenti principali saranno:

\begin{itemize}
    \item \textbf{Porta ricevente MQTT}: si occupa di ricevere i messaggi da MQTT e di inoltrarli al sistema ad eventi;
    \item \textbf{Interfaccia REST}: si occupa di ricevere le richieste degli utenti e di inoltrarle al sistema ad eventi;
    \item \textbf{Pila di payload}: É una pila contenente i payload di dati da analizzare;
    \item \textbf{Algoritmo di analisi}: Dato un payload si occupa di analizzarlo e di generare un nuovo stato;
\end{itemize}

La pila di Payload usa il paradigma del \textbf{Producer-Consumer}, in cui il produttore è la porta ricevente, mentre il consumatore è l'algoritmo di analisi.

É poi presente un database\footnote{Vedi sezione relativo al database e alla sua progettazione \ref{cap:db-sistema-coordinatore}} al quale il sistema si connette, per mantenere gli stati anche in caso di reboot del sistema.

Inoltre, se il programma farà comunque del caching dei dati, sarà comunque nel database che verranno depositati i dati quando la cache diventa troppo grande.

\section{Architettura del sistema}

%TODO: Michele

