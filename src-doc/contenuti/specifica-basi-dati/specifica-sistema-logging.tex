\section{Base di dati sistema logging}

\subsection{Abstract}

L’obiettivo è sviluppare un servizio che tenga traccia in maniera intelligente dei log. Questo servizio offre numerosi benefici legati alla retrospettiva. Il miglioramento delle prestazioni del sistema, la risoluzione rapida dei problemi, il rilevamento delle minacce alla sicurezza e la generazione di informazioni utili per l'analisi e il miglioramento continuo sono solo alcuni dei benefici che questo servizio permette.

\subsection{Analisi dei requisiti}

\subsubsection{Descrizione testuale}

Il microservizio per funzionare correttamente avrà bisogno di memorizzare ogni cambiamento, denominato {\it{log}}, a lungo termine. Dei {\it{log}} ci interessa:

\begin{itemize}
    \item l'id del misuratore che ha cambiato stato;
    \item l'istante in cui la modifica è  stata effettuata;
    \item il valore del log, che cambia a seconda che la modifica avvenga ad un sensore oppure ad un lampione;
    \item la tipologia di log.
\end{itemize}

\subsubsection{Glossario dei termini}

Per evitare ambiguità relative alle terminologie utilizzate è stato creato un documento denominato \textit{Glossario}.

Questo documento contiene tutti i termini specifici di settore utilizzati nei documenti, con le relative definizioni.

\subsubsection{Operazioni tipiche}

Le operazioni tipiche che ci si aspetta di avere sono:

\begin{center}
    \begin{tabularx}{\textwidth}{|l|X|}
        \hline
        \rowcolor{gray!30}
        \multicolumn{2}{|c|}{\textbf{OPERAZIONI TIPICHE}}
        \\
        \hline
        \rowcolor{gray!30}
        \textbf{{DESCRIZIONE}} & \textbf{{FREQUENZA D'USO}} \\
        \hline
        Lettura di molteplici log aggregati per l'estrazione dei trend & Molte volte \\
        \hline
        Scrittura del log & Molte volte\\
        \hline
    \end{tabularx}
\end{center}

\subsection{Progettazione concettuale}

\subsubsection{Analisi delle entità}

\textbf{Se non specificato l'attributo è NOT NULL}

%LOG
\begin{center}
    \begin{tabularx}{\textwidth}{|l|l|l|X|}
        \hline
        \rowcolor{gray!30}
        \multicolumn{4}{|c|}{\textbf{LOG}}\\
        \hline
        idMisuratore & INTEGER & Identifica univocamente, insieme a {\it{time}}, un log & Chiave\\
        \hline
        istanteModifica & LONG & Indica quando c'è stata una variazione ad un lampione o ad un sensore & Chiave\\
        \hline
        valore & INTEGER & \multicolumn{2}{l|}{Nel caso di un sensore, indica se è in grado di rilevare persone oppure no} \\
        & & \multicolumn{2}{l|}{Nel caso di un lampione, indica il suo valore di luminosità} \\
        \hline
        tipo & VARCHAR(50) & \multicolumn{2}{l|}{Indica se il sistema sta facendo riferimento ad un sensore oppure ad un lampione} \\
        \hline
    \end{tabularx}
\end{center}

\subsubsection{Schema ER concettuale}

\begin{center}
    \includegraphics[width=4.5cm]{contenuti/specifica-basi-dati/img-sbd/logging_concettuale.png}
\end{center}

\subsection{Progettazione logica}