\section{Base di dati sistema logging}

\subsection{Abstract}

L’obiettivo è sviluppare un servizio che tenga traccia in maniera intelligente dei log. Questo servizio offre numerosi benefici legati alla retrospettiva. Il miglioramento delle prestazioni del sistema, la risoluzione rapida dei problemi, il rilevamento delle minacce alla sicurezza e la generazione di informazioni utili per l'analisi e il miglioramento continuo sono solo alcuni dei benefici che questo servizio permette.

\subsection{Analisi dei requisiti}

\subsection{Progettazione concettuale}

\subsubsection{Analisi delle entità}

\textbf{Se non specificato l'attributo è NOT NULL}

%LOG
\begin{center}
    \begin{tabularx}{\textwidth}{|l|l|l|X|}
        \hline
        \rowcolor{gray!30}
        \multicolumn{4}{|c|}{\textbf{LOG}}\\
        \hline
        idMisuratore & INTEGER & Identifica univocamente, insieme a {\it{time}}, un log & Chiave\\
        \hline
        time & LONG & Indica quando c'è stata una variazione ad un lampione o ad un sensore & Chiave\\
        \hline
        valore & INTEGER & \multicolumn{2}{l|}{Nel caso di un sensore, indica se è in grado di rilevare persone oppure no} \\
        & & \multicolumn{2}{l|}{Nel caso di un lampione, indica il suo valore di luminosità} \\
        \hline
        tipo & VARCHAR(50) & \multicolumn{2}{l|}{Indica se il sistema sta facendo riferimento ad un sensore oppure ad un lampione} \\
        \hline
    \end{tabularx}
\end{center}

\subsubsection{Schema ER concettuale}

\subsection{Progettazione logica}