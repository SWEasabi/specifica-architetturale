\section{Base di dati sistema coordinazione}\label{sec:sbd-sistema-coordinazione}

\subsection{Abstract}

L'obiettivo è quello di sviluppare un sistema che coordini e decida quando illuminare o non illuminare specifiche zone. Questo sistema dovrà tenere traccia di tutte le informazioni relative allo stato attuale dei lampioni.

Il sistema si collegherà ad altre componenti esterne, utilizzando algoritmi proprietari, e deciderà quando effettuare cambiamenti di stato. Si consiglia la visione del capitolo \ref{cap:sistema-coordinazione}.

\subsection{Analisi dei requisiti}

\subsubsection{Descrizione testuale}

Il microservizio per funzionare correttamente avrà bisogno di salvare lo stato dei lampioni per controllare che non vengano fatte modifiche con tempistiche troppo ravvicinate, e inoltre vengono memorizzati la luminosità e l'istante dell'ultima modifica di un lampione. 

\subsubsection{Glossario dei termini}

Per evitare ambiguità relative alle terminologie utilizzate è stato creato un documento denominato \textit{Glossario}.

Questo documento contiene tutti i termini specifici di settore utilizzati nei documenti, con le relative definizioni.

\subsubsection{Operazioni tipiche}

Le operazioni tipiche che ci si aspetta di avere sono:

\begin{center}
    \begin{tabularx}{\textwidth}{|l|X|}
        \hline
        \rowcolor{gray!30}
        \multicolumn{2}{|c|}{\textbf{OPERAZIONI TIPICHE}}
        \\
        \hline
        \rowcolor{gray!30}
        \textbf{{DESCRIZIONE}} & \textbf{{FREQUENZA D'USO}} \\
        \hline
        Lettura dello stato in cui si trova il lampione & Molte volte \\
        \hline
        Scrittura dello stato di un lampione & Molte volte\\
        \hline
    \end{tabularx}
\end{center}

\subsection{Progettazione concettuale e progettazione logica}

In seguito alla riunione dei progettisti del 30/07\footnote{vedasi verbale VIN\_20230730}, il sistema di coordinazione non avrà un database "indipendente" ma utilizzerà, a seconda dell'operazione richiesta, il database del sistema di anagrafe e quello del sistema di logging.