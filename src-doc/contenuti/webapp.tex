\chapter{Webapp}\label{cap:webapp}

\section{Scopo del sistema}

La webapp permette all'utente di avere un interfaccia grafica con la quale gestire e monitorare lampioni,sensori ed aree inoltre permette di visualizzarne le informazioni relative.

\section{Requisiti coperti dal sistema}

\subsection{RV\_01}
RV\_01: l'applicazione deve essere visualizzabile su dispositivi mobile.

\subsection{RV\_02}
RV\_02: l'applicazione client deve poter essere utilizzata sulla versione più recente di Chrome (v. 110.0).

\subsection{RV\_03}
RV\_03: l'applicazione client deve poter essere utilizzata sulla versione più recente di Firefox (v. 110.0).

\subsection{RV\_04}
RV\_04: l'applicazione client deve poter essere utilizzata sulla versione più recente di Safari (v. 16.3).

\subsection{RV\_05}
RV\_05: l'applicazione client deve essere conforme almeno al livello AA delle WCAG.


\section{Descrizione del sistema}

La webapp è stata sviluppata utilizzando il framework AngularJS, il quale permette di creare applicazioni web single-page e di gestire le dipendenze tra le varie componenti dell'applicazione. Inoltre, è stato utilizzato il framework Bootstrap per la gestione della parte grafica. L'applicazione utilizza componenti di Angular, servizi e chiamate API per comunicare con il server. Le principali features dell'applicazione sono:
\begin{itemize}
    \item Visualizzazione lista di lampioni, sensori ed aree;
    \item Cambiare lo status dei lampioni e dei sensori;
    \item Aggiungere lampioni,sensori ed aree;
    \item Gestire caricamenti ed errori durante le chiamate API;
    \item Implementare l'autenticazione;
    \item Proteggere i collegamenti all'applicazione; 
\end{itemize} 

\section{Architettura del sistema}

La classe Model implementa tre interfacce che definiscono lo status di lampioni,sensori ed aree, queste definiscono la struttura e forniscono le proprietà per immagazinare informazioni quali status,identificativo e alias.
Le tre interfacce sono:
\begin{itemize}
    \item \textbf{LampStatus:} definisce lo stato di un lampione;
    \item \textbf{SensorStatus:} definisce lo stato di un sensore;
    \item \textbf{AreaStatus:} definisce lo stato di un'area;
\end{itemize}

La classe \textbf{Apiservice} definisce i metodi per effettuare le chiamate API e interagire con il server, questo servizio mette a dispozione i metodi per:
\begin{itemize}
    \item Ottenere la lista di lampioni, sensori ed aree;
    \item Ottenere lo stato di un lampione, sensore o area;
    \item Modificare lo stato di un lampione, sensore o area;
    \item Aggiungere un lampione, sensore o area;
    \item Ottenere i dati di un sensore;
    \item Ottenere i dati di un'area;
\end{itemize}

I componenti utilizzati dall'applicazione sono:
\begin{itemize}
    \item \textbf{LampsListComponent:} visualizza la lista di lampioni;
    \item \textbf{SensorsListComponent:} visualizza la lista di sensori;
    \item \textbf{AreasListComponent:} visualizza la lista di aree;
    \item \textbf{LampButtonComponent:} rappresenta un lampione della lista con il proprio stato e permette di modificarlo;
    \item \textbf{SensorButtonComponent:} rappresenta un sensore della lista con il proprio stato e permette di modificarlo;
    \item \textbf{AreaButtonComponent:} rappresenta un'area della lista con il proprio stato e permette di modificarlo;
\end{itemize}

L'applicazione presenta delle classi per l'utilizzo di altri microservizi:
\begin{itemize}
    \item \textbf{AuthService:} controlla l'autenticazione dell'utente;
    \item \textbf{AppService:} gestisce lo stato dell'applicazione mantenedo lo status di lampioni, sensori ed aree, questo servizio fornisce i metodi per l'aggiunta di nuovi dispositivi e comunica con l'API per ottenere dati,aggiornare gli status e gestire caricamenti ed errori;
\end{itemize}

La classe \textbf{AuthGuard} permette di proteggere i collegamenti all'applicazione, in questo modo l'utente non autenticato non può accedere alle pagine dell'applicazione.